% Abstract for the TUM report document
% Included by MAIN.TEX


\cleardoublepage




\vspace*{2cm}
\begin{center}
{\Large \bf Abstract}
\end{center}
\vspace{1cm}

With the advent of new RGBD sensors and their deployment in robotic systems, it is now possible to get a frame-to-frame extraction and tracking of human poses in the robot's environment. By reasoning on sequences of such poses one can learn and recognize different human activities. 

The subject of this work is to devise a practical method capable of online (incremental) activity recognition. Three different approaches are taken. 
The first one is an implementation of an existing model which extracts representative poses from 
an activity and uses \textit{Support Vector Machines} for real-time activity recognition by classifying the distribution of these poses. Several extensions are proposed, such as using sequence similarity measures for the classification. An integration into the \textit{Robot Operating System} is developed. This results in a module capable of real-time classification but not capable of online activity recognition.

The second approach exploits the \textit{Discriminative Sequence Back-Constraint GP-LVM}, which is a model that learns a non-linear lower-dimensional manifold for the poses, and optimizes this latent space in such a way, that the similarity of the sequences and the discriminative properties of the activity classes are captured. Online activity recognition is performed by classifying the centroid of the latent clusters belonging to the individual activity sequences. Tests on the Cornell Daily Living Activity data set reveal that this model is not capable of finding a latent space which captures all activities, because of the hard optimization problem. 

The third approach is a novel method for online activity recognition based on separately learning a dense motion flow field for each activity class in its lower-dimensional representation through several \textit{Gaussian Process Regressions}. Online recognition is done by incrementally computing and accumulating the probability that the actual pose movement corresponds to each learned flow field. The issue of finding a smooth inverse mapping from observed space to latent space is discussed and some alternatives for the dimensionality reduction are proposed.
